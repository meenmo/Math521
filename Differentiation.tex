% --------------------------------------------------------------
% This is all preamble stuff that you don't have to worry about.
% Head down to where it says "Start here"
% --------------------------------------------------------------
 
\documentclass[12pt]{article}
 
\usepackage[nouppercase,headsepline,footsepline,plainfootsepline]{scrpage2}
\automark{section}
\pagestyle{scrheadings}
%\clearscrheadfoot
\ihead{Math 521\;\; Differentiation}
%\ofoot[\pagemark]{\pagemark}% Optional argument controls chapter-starting pages
\ifoot[(Author)]{{\sl \hfill Meenmo K.}}

\usepackage[margin=1in]{geometry} 
\usepackage{amsmath,amsthm,amssymb,scrextend}
\usepackage{fancyhdr}
\usepackage{enumitem}
\usepackage{amsmath}
\usepackage{amssymb}
\usepackage{textcomp}
\usepackage{fancybox}
\usepackage{tikz}
\usepackage{cancel}
\usepackage{tasks}


\newcommand{\N}{\mathbb{N}}
\newcommand{\Z}{\mathbb{Z}}
\newcommand{\I}{\mathbb{I}}
\newcommand{\R}{\mathbb{R}}
\newcommand{\Q}{\mathbb{Q}}
\renewcommand{\qed}{\hfill$\blacksquare$}
\let\newproof\proof
\renewenvironment{proof}{\begin{addmargin}[1em]{0em}\begin{newproof}}{\end{newproof}\end{addmargin}\qed}
% \newcommand{\expl}[1]{\text{\hfill[#1]}$}
\setlength{\parindent}{0pt}
\newenvironment{theorem}[2][Theorem]{\begin{trivlist}
\item[\hskip \labelsep {\bfseries #1}\hskip \labelsep {\bfseries #2.}]}{\end{trivlist}}
\newenvironment{lemma}[2][Lemma]{\begin{trivlist}
\item[\hskip \labelsep {\bfseries #1}\hskip \labelsep {\bfseries #2.}]}{\end{trivlist}}
\newenvironment{problem}[2][Problem]{\begin{trivlist}
\item[\hskip \labelsep {\bfseries #1}\hskip \labelsep {\bfseries #2.}]}{\end{trivlist}}
\newenvironment{exercise}[2][Exercise]{\begin{trivlist}
\item[\hskip \labelsep {\bfseries #1}\hskip \labelsep {\bfseries #2.}]}{\end{trivlist}}
\newenvironment{reflection}[2][Reflection]{\begin{trivlist}
\item[\hskip \labelsep {\bfseries #1}\hskip \labelsep {\bfseries #2.}]}{\end{trivlist}}
\newenvironment{proposition}[2][Proposition]{\begin{trivlist}
\item[\hskip \labelsep {\bfseries #1}\hskip \labelsep {\bfseries #2.}]}{\end{trivlist}}
\newenvironment{corollary}[2][Corollary]{\begin{trivlist}
\item[\hskip \labelsep {\bfseries #1}\hskip \labelsep {\bfseries #2.}]}{\end{trivlist}}
 
 
\begin{document}
\section{Differentiation}
\subsection{Differentiable Functions}
\begin{block}{\bf Def 5.1}
\begin{itemize}
\item Let $f\colon(a,b)\to \mathbb{R},\; a<b\in\overline{\mathbb{R}}$. $f$ is {\sl differentiable at $x\in(a,b)$} if $\lim\limits_{h\to0}\frac{f(x+h)-f(x)}{h}$ exists. 

\item If this limit exists, 
$$f'(x)=\lim\limits_{h\to 0}\frac{f(x+h)-f(x)}{h}$$ is the {\sl derivative of $f$ at $x$}

\item If $f$ is differentiable at all $x\in(a,b),$ we say $f$ is {\sl differentiable}.
\end{itemize}
\end{block}

\vspace{1\baselineskip}
\begin{block}{\bf Theorem 5.2} Let $f\colon (a,b)\to \mathbb{R}$ be differentiable at $x\in(a,b).$ Then $f$ is continuous at $x$.\end{block}

\vspace{1\baselineskip}

\begin{block}{\sl Proof.}
\begin{itemize}
    \item Let $x_n\to x,$ i.e. $\exists\;$ a sequence $h_n\to 0$ such that $x_n=x+h_m$. Then 
    $$f(x_n)-f(x) =\frac{f(x+h_n)-f(x)}{h_n}\cdot h_n \longrightarrow f'(x)\cdot 0 = 0$$
\end{itemize}
\end{block}

\vspace{1\baselineskip}
\begin{block}{\sl Remark 5.3}
\begin{itemize}
    \item $f$ is differentiable $\Rightarrow\;f$ is continuous.
    \item $f$ is continuous $\nRightarrow\;f$ is differentiable
    \item $\exists\;f\colon\mathbb{R}\to\mathbb{R}$ is continuous such that $f $is not differentiable at any $x\in\mathbb{R}$.
\end{itemize}
\end{block}

\vspace{1\baselineskip}
\begin{block}{\bf Theorem 5.4}Let $f,g\colon (a,b)\rightarrow\mathbb{R},$ differentiable at $x\in(a,b).$ Then
\begin{enumerate}[label=(\roman*)]
    \item $(f+g)(x)$ is differentiable at $x,\; (f+g)'(x)=f'(x)+g'(x).$
    \item $(fg)$ is differentiable at $x,\; (fg)'(x)=f'(x)g(x)+g'(x)f(x).$
    \item $\left(\frac{f}{g}\right)$ is differentiable at $x$ if $g(x)\neq 0,$ and $\left(\frac{f}{g}\right)'(x)=\frac{g(x)f'(x)-f(x)g'(x)}{g(x')^2}$
\end{enumerate}
\end{block}

\vspace{1\baselineskip}
\begin{block}{\sl Proof.}
\begin{enumerate}[label=(\roman*)]
    \item Exercise
    \item Consider $$\frac{(fg)(x+h)-(fg)(x)}{h} = \frac{f(x_h)-f(x))g(x_h)+f(x)(g(x+h)-g(x))}{h}$$
    Note $$\frac{f(x+h)-f(x)}{h}\rightarrow f'(x)\text{as h goes to inf}$$
    $$\frac{g(x+h)-g(x)}{h}\to g(x)\text{as h to 0}$$
    $$g(x+h) \to g(x)\text{as $h\to 0$ as $g$ is continuous at $x$.}$$
    So $$\frac{(fg)(x_h)-(fg)(x)}{h}\to f'(x)g(x)+g(x)g'(x).$$
    
    \item $$\frac{\left(\frac{f}{g}\right)(x+h)-\left(\frac{f}{g}\right)(x)}{h} = \frac{f(x+h)g(x)-f(x)g(x+h)}{hg(x+h)g(x)}$$
    $$=\frac{(f(x+h)-f(x))g(x)+f(x)(g(x)-g(x+h))}{h\cdot g(x+h)g(x)}$$
    $$\longrightarrow \frac{f'(x)g(x)-f(x)g'(x)}{g^2(x)}$$
\end{enumerate}
\end{block}

\vspace{1\baselineskip}
\begin{block}{\bf Example}
\begin{enumerate}[label=(\roman*)]
    \item Let $f(x)=c\in \mathbb{R}.$ Then $\frac{f(x+h)-f(x)}{h} = \frac{c-c}{h}=0\rightarrow 0.$ So $f'(x)=0$.
    \item Let $g(x)=x.$ Then $\frac{g(x+h)-g(x)}{h} = \frac{x+h-x}{h} = 1 \rightarrow 1$ (as $h\to 0$). So $g'(x)=1$.
    \item Suppose we know $\frac{d}{dx} x^n = nx^{n-1}.$ 
    \\Check $\frac{d}{dx}(x^{n+1}) = \frac{d}{dx} (x\cdot x^n) = 1\cdot x^n+xnx^{n-1}=(n+1)x^n.$
\end{enumerate}
\end{block}

\vspace{1\baselineskip}
\begin{block}{\bf Example}\\ 
Image
\end{block}

\vspace{1\baselineskip}
\begin{block}{\bf Example}\\ 

\end{block}

\vspace{1\baselineskip}
\begin{block}{\bf Theorem}
Let $f\colon(alb)\rightarrow\mathbb{R},\; f(x)\in(c,d),$ be differentiable at $x\in(a,b)$. Let $g\colon (c,d)\rightarrow\mathbb{R}$ be differentiable at $f(x).$ Then $g\circ f$ is differentiable at $x$ and $(g\circ x)=g'(f(x))f'(x)$ 
\end{block}
\end{document}