% --------------------------------------------------------------
% This is all preamble stuff that you don't have to worry about.
% Head down to where it says "Start here"
% --------------------------------------------------------------
 
\documentclass[12pt]{article}
 
\usepackage[margin=1in]{geometry} 
\usepackage{amsmath,amsthm,amssymb,scrextend}
\usepackage{fancyhdr}
\usepackage{enumitem}
\usepackage{amsmath}
\usepackage{amssymb}
\usepackage{textcomp}
\usepackage{fancybox}
\usepackage{tikz}
\pagestyle{fancy}

 
\newcommand{\N}{\mathbb{N}}
\newcommand{\Z}{\mathbb{Z}}
\newcommand{\I}{\mathbb{I}}
\newcommand{\R}{\mathbb{R}}
\newcommand{\Q}{\mathbb{Q}}
\renewcommand{\qed}{\hfill$\blacksquare$}
\let\newproof\proof
\renewenvironment{proof}{\begin{addmargin}[1em]{0em}\begin{newproof}}{\end{newproof}\end{addmargin}\qed}
% \newcommand{\expl}[1]{\text{\hfill[#1]}$}
 
\newenvironment{theorem}[2][Theorem]{\begin{trivlist}
\item[\hskip \labelsep {\bfseries #1}\hskip \labelsep {\bfseries #2.}]}{\end{trivlist}}
\newenvironment{lemma}[2][Lemma]{\begin{trivlist}
\item[\hskip \labelsep {\bfseries #1}\hskip \labelsep {\bfseries #2.}]}{\end{trivlist}}
\newenvironment{problem}[2][Problem]{\begin{trivlist}
\item[\hskip \labelsep {\bfseries #1}\hskip \labelsep {\bfseries #2.}]}{\end{trivlist}}
\newenvironment{exercise}[2][Exercise]{\begin{trivlist}
\item[\hskip \labelsep {\bfseries #1}\hskip \labelsep {\bfseries #2.}]}{\end{trivlist}}
\newenvironment{reflection}[2][Reflection]{\begin{trivlist}
\item[\hskip \labelsep {\bfseries #1}\hskip \labelsep {\bfseries #2.}]}{\end{trivlist}}
\newenvironment{proposition}[2][Proposition]{\begin{trivlist}
\item[\hskip \labelsep {\bfseries #1}\hskip \labelsep {\bfseries #2.}]}{\end{trivlist}}
\newenvironment{corollary}[2][Corollary]{\begin{trivlist}
\item[\hskip \labelsep {\bfseries #1}\hskip \labelsep {\bfseries #2.}]}{\end{trivlist}}
 
\begin{document}
 
% --------------------------------------------------------------
%                         Start here
% --------------------------------------------------------------

\lhead{Math 521}
\chead{Homework 2}
\rhead{Meenmo Kang}

\noindent
\textbf{Question 2.1}\\
A complex number $\z \in \;\mathbb{C}$ is said to be algebraic if there exists integers $a_0,...,a_n$ such that $$a_0z^n + a_1z^{n-1} + ... + a_{n-1}z + a_0 = 0$$
Prove that the set of all algebraic numbers is countable. Hint: For every positive integer N,
there are only finitely many equations with
$$n + |a_0| + |a_1| + ... + |a_n| = N.$$
\begin{itemize}
    \item Suppose $S_N$ be the set of equations that are equal to a positive integer N.
    \item Since there are only finitely many equations for a single N, as stated above, the collection of these equations, $S_N$, is at most countable. 
    \item For each algebraic number $a_iz^{n-i}$, where $i_{\mathbb{Z}\ge 0}$, an equation for $S_N$ can be formed for some N.
    \item Thus the set of all algebraic number is countable.\\
\end{itemize}

\noindent
\textbf{Question 2.2}\\
Prove that there exist real numbers that are not algebraic.
\begin{itemize}
    \item Suppose not. Assume there exists real numbers that are algebraic.
    \item A set of algebraic number is countable, as we proved on the Question 2.1.
    \item Hence the set of those real numbers is countable.  
    \item This contradicts the statement we prove.\\
    \end{itemize}

\noindent
\textbf{Question 2.3}\\
Is the set of all irrational numbers countable?
\begin{itemize}
    \item No. 
    
    \\Because the set of all irrational numbers cannot be 1-1 paired with any countable set such as $\mathbb{Z}.$\\
    
    Or if the set of all irrational numbers is countable, then the set of all real number consists of irrational real and rational real numbers is countable, which is impossible.
\end{itemize}

\newpage
\noindent
\textbf{Question 2.4}\\
Prove that each of the following distance functions $d: \mathbb{R}^2$ x $\mathbb{R}^2 \rightarrow \mathbb{R}$ on $\mathbb{R}^2$ define a metric.\\

\noindent
Recall the definition of metric space.
\begin{enumerate}[label=(\roman*)]
    \item d(p,q) $>$ 0 if $p\neq q$; d(p,p) = 0
    \item d(p,q) = d(q,p)
    \item d(p,q) $\le$ d(p,r) + d(r,p) \;\;$\forall\; r\in \mathbb{R}$.\\
\end{enumerate}

 
    $$
        d(x,y) = 
        \begin{cases}
            |x-y| \qquad \text{if $x = \lambda y$, some $\lambda \; \in \mathbb{R}$}\\
            |x| + |y| \quad\;\; \text{if $x \neq \lambda y$, for any $\lambda\;\in\mathbb{R}$}.\\
        \end{cases}
    $$
    
    \begin{itemize}
        \item \text{if $x = \lambda y$}
            \begin{enumerate}[label=(\roman*)]
                \item By the definition of absolute value, $|x - y| \ge 0$\\
                \Hence $d(x,y) = |x - y| > 0$ only and only if $x \neq y$.\\
                Otherwise, in case of $x = y, d(x,y) = |x- y| = 0.$\\
                \item $d(x,y) = |x - y| = |y - x| = d(y,x)$
                \item $d(x,y) \le d(x,z) + d(z,y)\; \Leftrightarrow \; |x - y| \le |x - z| + |z - y|\;\;
                \forall\; z \in X.$\\
            \end{enumerate}
            
        \item \text{if $x \neq \lambda y$}
            \begin{enumerate}[label=(\roman*)]
                \item Then d(x,y) = $|x| + |y| > 0$
                \item d(x,y) = $|x| + |y| = |y| + |x|$ = d(y,x)
                \item d(x,y) = $|x| + |y| \le |x| + |z| + |z| + |y|\;\; \forall z \in \mathbb{R}$
            \end{enumerate}
    \end{itemize}
    \indent
        \quad\; Thus this is a metric space.\\
    
    
\newpage
 Writing x = $(x_1,x_2),$ y = $(y_1,y_2)$,
    $$
        d(x,y) = 
        \begin{cases}
            |x_2-y_2| \qquad\qquad\qquad\;\;\; \text{if}\;\; $x_1 = y_1$\\
            |x_2| + |y_1 - x_1| +|y_2| \quad\;\; \text{if $x_1 \neq y_1$}
        \end{cases}
    $$

\\

    \begin{itemize}
        \item \text{if $x_1 = y_1$}
            \begin{enumerate}[label=(\roman*)]
                \item d(x,y) = $|x_2 - y_2| > 0$ if and only if $x_2 \neq y_2$\
                d(x,y) = 0 \qquad\qquad\quad if and only if $x_2 = y_2$
                \item d(x,y) = $|x_2 - y_2| = |-1|\cdot |y_2 - x_2| = |y_2 - x_2|$ = d(y,x) 
                \item d(x,y) = $|x_2 - y_2| = |x_2 - z_2 + z_2 - y_2| \le |x_2 - z_2| + |z_2 + y_2|\;\;\forall z_2 \in \mathbb{R}$
            \end{enumerate}
            
        \item \text{if $x_1 \neq y_1$}
            \begin{enumerate}[label=(\roman*)]
               \item d(x,y) = $|x_2| + |y_1 - x_1| +|y_2| > |y_1 - x_1| > 0$\\
               d(x,y) = 0 is not the case for this condition since $x_1 \neq y_1$
               \item if d(x,y) = $|x_2| + |y_1 - x_1| +|y_2| = |y_2| + |x_1 - y_1| + |x_2| = |y_2| + |-1|\cdot |-x_1 + y_1| + |x_2|$\\
               = $|y_2| + |y_1 - x_1| + |x_2| =$ d(y,x)
               \item  
               \begin{itemize}
                   \item if $z_1 = x_1$ (or $z_1 = y_1$)\\
                   d(x,z) + d(z,y) = $|x_2| + |y_1 - z_1| + |x_1 - z_1| + |y_2| \\
                   = |x_2| + |y_1 - z_1| + |z_1 - x_2| + |y_2|\\
                   \ge  |x_2| + |y_1 - z_1| + |x_2 - z_2 + y_2|\\ 
                   = |x_2| + |y_1 - z_1| + |y_2|$\\
                   
                   \item if $x_1 \neq y_1 \neq z_1$\\
                   d(x,z) + d(z,y) = $|x_2| + |z_1 - x_1| + |z_2| + |z_2| + |y_1 - z_1| + |y_2| \\
                   \ge |x_2| + |y_1 - z_1| + |z_1 - x_1| + |y_2|\\
                   \ge |x_2| + |y_1 - z_1 + z_1 - x_1| + |y_2| = |x_2| + |y_1 - x_1| + |y_2|\\ = $d(x,y)
               \end{itemize}
              

            \end{enumerate}
    \end{itemize}


\newpage
\noindent
\textbf{Question 2.5}\\
For each of the two metrics in Q2.4, find
\begin{enumerate}[label=(\roman*)]
    \item $B_{1/2}((1,1))$, that is the ball of radius 1/2 centered at (1,1).
    
    $\Rightarrow B_{\frac{1}{2}} = \{(x,y) \in \mathbb{R}^2\;|\; d\{(1,1), (x,y)\} < \frac{1}{2}\}$
    
    \begin{itemize}
        \item 
            $
        d(x,y) = 
        \begin{cases}
            |x-y| \qquad \text{if $x = \lambda y$, some $\lambda \; \in \mathbb{R}$}\\
            |x| + |y| \quad\;\; \text{if $x \neq \lambda y$, for any $\lambda\;\in\mathbb{R}$}.

        \end{cases}
        $\\
        
        
        \begin{itemize}
            \item \textbf{Case 1}: (x,y) = $\lambda(1,1)$\\
            
            d((1,1),(x,y)) = $|(1,1) - (x,y)|\\
            = \sqrt{(x-1)^2 + (y-1)^2} < \frac{1}{2}$\\
            $\Leftrightarrow (x-1)^2 + (y-1)^2 < \frac{1}{4}$\\
            
            \item \textbf{Case 2}: (x,y) $\neq \lambda(1,1)$\\
            
            d((0,1),(x,y)) = $|(1,1)| + |(x,y)|$\\
            $= \sqrt{2} + \sqrt{x^2 + y^2} < \frac{1}{2} \Leftrightarrow \sqrt{x^2 + y^2} < \frac{1}{2} - \sqrt{2}$\\
            $\Leftrightarrow x^2 + y^2 < (\frac{1}{2} - \sqrt{2})^2$\\
            \\
            
        \end{itemize}\\
        
        
    \item 
    $
        d(x,y) = 
        \begin{cases}
            |x_2-y_2| \qquad\qquad\qquad\;\;\; \text{if}\;\; $x_1 = y_1$\\
            |x_2| + |y_1 - x_1| +|y_2| \quad\;\; \text{if $x_1 \neq y_1$}
        \end{cases}
    $
    
    \begin{itemize}
        \item \textbf{Case 1}: $x_1 = 1$\\
        
        d((1,1),(x,y)) = $|y-1| < \frac{1}{2}\\ 
        \;\Leftrightarrow\; -\frac{1}{2} < y - 1 < \frac{1}{2}$
        $\; \Leftrightarrow\; \frac{1}{2} < y < \frac{3}{2}$\\
        
        \item \textbf{Case 2}: $x_1 \neq 1$\\
        
        d((1,1), (x,y)) = $|y| + |x-1| + |1| < \frac{1}{2}$\\
        $|y| + |x-1| < -\frac{1}{2}$, which is impossible.
    \end{itemize}
    
    \end{itemize}
    
    \newpage
    \item $B_{2}\;\;((1,1))$, that is the ball of radius 2 centered at (1,1).
    
    $\Rightarrow B_{2}(1,1) = \{(x,y) \in \mathbb{R}^2\;|\; d\{(1,1), (x,y)\} < 2\}\\$
    
    \begin{itemize}
        \item 
            $
        d(x,y) = 
        \begin{cases}
            |x-y| \qquad \text{if $x = \lambda y$, some $\lambda \; \in \mathbb{R}$}\\
            |x| + |y| \quad\;\; \text{if $x \neq \lambda y$, for any $\lambda\;\in\mathbb{R}$}.

        \end{cases}
        $\\
        
        
        \begin{itemize}
            \item \textbf{Case 1}: (x,y) = $\lambda(1,1)$\\
            
            d((1,1),(x,y)) = $|(1,1) - (x,y)|\\
            = \sqrt{(x-1)^2 + (y-1)^2} < 2$\\
            $\Leftrightarrow (x-1)^2 + (y-1)^2 < 4$\\
            
            \item \textbf{Case 2}: (x,y) $\neq \lambda(1,1)$\\
            
            d((0,1),(x,y)) = $|(1,1)| + |(x,y)|$\\
            $= \sqrt{2} + \sqrt{x^2 + y^2} < 2 
            \Leftrightarrow \sqrt{x^2 + y^2} < 2 - \sqrt{2}$\\
            $\Leftrightarrow x^2 + y^2 < (2 - \sqrt{2})^2$\\
            \\
            
        \end{itemize}\\
        
        
    \item 
    $
        d(x,y) = 
        \begin{cases}
            |x_2-y_2| \qquad\qquad\qquad\;\;\; \text{if}\;\; $x_1 = y_1$\\
            |x_2| + |y_1 - x_1| +|y_2| \quad\;\; \text{if $x_1 \neq y_1$}
        \end{cases}
    $
    
    \begin{itemize}
        \item \textbf{Case 1}: $x_1 = 1$\\
        
        d((1,1),(x,y)) = $|y-1| < 2\\ 
        \;\Leftrightarrow\; -2 < y - 1 < 2$
        $\; \Leftrightarrow\; -1 < y < 3$\\
        
        \item \textbf{Case 2}: $x_1 \neq 1$\\
        
        d((1,1), (x,y)) = $|y| + |x-1| + |1| < 2$\\
        $|y| + |x-1| < 1$
    \end{itemize}
    
    \end{itemize}
\end{enumerate}

\newpage
\noindent
\textbf{Question 2.6}\\
 Prove that the set $\{(x,y) \in \mathbb{R}^2\;|\;y < |x|\}$ is open with respect to the Euclidean metric.\\
 
\begin{itemize}
    \item \textbf{Def}: Open Set\\
    A subset $A$ of $R^n$ is said to be \textbf{open} (in $R^n$) if and only if for every $x \in A\;\; \exists\; \epsilon > 0$ such that $B_\epsilon (x) \subseteq A$.
\end{itemize}




 
 
\end{document}