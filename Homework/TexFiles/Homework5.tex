% --------------------------------------------------------------
% This is all preamble stuff that you don't have to worry about.
% Head down to where it says "Start here"
% --------------------------------------------------------------
 
\documentclass[12pt]{article}
 
\usepackage[margin=1in]{geometry} 
\usepackage{amsmath,amsthm,amssymb,scrextend}
\usepackage{fancyhdr}
\usepackage{enumitem}
\usepackage{amsmath}
\usepackage{amssymb}
\usepackage{textcomp}
\usepackage{fancybox}
\usepackage{tikz}
\usepackage{cancel}
\usepackage{tasks}
\pagestyle{fancy}


\newcommand{\N}{\mathbb{N}}
\newcommand{\Z}{\mathbb{Z}}
\newcommand{\I}{\mathbb{I}}
\newcommand{\R}{\mathbb{R}}
\newcommand{\Q}{\mathbb{Q}}
\renewcommand{\qed}{\hfill$\blacksquare$}
\let\newproof\proof
\renewenvironment{proof}{\begin{addmargin}[1em]{0em}\begin{newproof}}{\end{newproof}\end{addmargin}\qed}
% \newcommand{\expl}[1]{\text{\hfill[#1]}$}
\setlength{\parindent}{0pt}
\newenvironment{theorem}[2][Theorem]{\begin{trivlist}
\item[\hskip \labelsep {\bfseries #1}\hskip \labelsep {\bfseries #2.}]}{\end{trivlist}}
\newenvironment{lemma}[2][Lemma]{\begin{trivlist}
\item[\hskip \labelsep {\bfseries #1}\hskip \labelsep {\bfseries #2.}]}{\end{trivlist}}
\newenvironment{problem}[2][Problem]{\begin{trivlist}
\item[\hskip \labelsep {\bfseries #1}\hskip \labelsep {\bfseries #2.}]}{\end{trivlist}}
\newenvironment{exercise}[2][Exercise]{\begin{trivlist}
\item[\hskip \labelsep {\bfseries #1}\hskip \labelsep {\bfseries #2.}]}{\end{trivlist}}
\newenvironment{reflection}[2][Reflection]{\begin{trivlist}
\item[\hskip \labelsep {\bfseries #1}\hskip \labelsep {\bfseries #2.}]}{\end{trivlist}}
\newenvironment{proposition}[2][Proposition]{\begin{trivlist}
\item[\hskip \labelsep {\bfseries #1}\hskip \labelsep {\bfseries #2.}]}{\end{trivlist}}
\newenvironment{corollary}[2][Corollary]{\begin{trivlist}
\item[\hskip \labelsep {\bfseries #1}\hskip \labelsep {\bfseries #2.}]}{\end{trivlist}}
 
 
\begin{document}
\settasks{
counter-format=(tsk[r]),
label-width=4ex
}
% --------------------------------------------------------------
%                         Start here
% --------------------------------------------------------------

\lhead{Math 521}
\chead{Homework 5}
\rhead{Meenmo Kang}


\textbf{Question 5.1}
\begin{enumerate}[label=(\roman*)]
    \item Let $\a,\; x\in [\infty,\infty)$. Suppose that for all $\epsilon>0,$ we have that $x \le a+\epsilon$. Prove that $x\le a$.
    \begin{itemize}
        \item Suppose by contradiction that $x > a$.
        \item Consider $\epsilon' > 0$ such that $x > a+\epsilon$, which contradicts to $x\le a+\epsilon'$.
        \item Thus $x\le a$.
    \end{itemize}
    
    \item Let $(x_n) \subset \mathbb{R}$ be such that $x_n \to x$ as $n \to\infty$ and, for all $n\in\mathbb{N},\; x_n \le a$. Prove that $x\le a$.

    \begin{itemize}
        \item Suppose by contradiction that $x>a$.
        \item Suppose, further, $\exists\;\epsilon>0$ such that $x-\epsilon>a$. 
        \item Since $x_n$ converges to $x$, $|x_n - x| <\epsilon$ if $\exists\; n$ such that $n\in\mathbb{N}.$
        \item i.e. $x-\epsilon < x_n < x+\epsilon\;\Leftrightarrow\; a < x-\epsilon<x$, which is a contradiction to $x>a$ that we made.
        \item Thus $x\le a$.
    \end{itemize}
    
    \item Suppose now $x_n< a$ for all $n\in \mathbb{N}$ and $x_n \rightarrow x$. Is it true that $x<a$? Give a proof or counterexample.
    \begin{itemize}
        \item Assume $x_n = -\frac{1}{n}$ and $a = 0$.
        \item $x_n$ is strictly lesser than 0, but $x_n$ converges to 0 which is $a$.
    \end{itemize}
    
\end{enumerate}

\vspace{1.5\baselineskip}

\textbf{Question 5.2}\\
Let $x_n = \sqrt{n}$ for all $n\in \mathbb{N}$.
\begin{enumerate}[label=(\roman*)]
    \item Prove $(x_{n+1} - x_n) \rightarrow 0$ as $n\rightarrow \infty$.
        $$\lim_{n\rightarrow\infty} \left\{ (\sqrt{n+1}-\sqrt{n}) 
        = (\sqrt{n+1}+\sqrt{n}) \cdot                 \frac{\sqrt{n+1}-\sqrt_n}{\sqrt{n+1}+\sqrt{n}} 
        = \frac{\cancel{n}+1 - \cancel{n}}{\sqrt{n+1}-\sqrt{n}} 
        = \frac{1}{\sqrt{n+1} - \sqrt{n}} \right\} = 0
        $$
        
    \item Is $(x_n)$ a Cauchy sequence? Prove your answer.
    \begin{itemize}
        \item Recall the definition of Cauchy sequences.\\
        $(x_n)$ is a Cauchy if for every $\epsilon>0$ there is an integer $N$ such that $d(p_n,p_m)<\epsilon$ if $n\ge N$ and $m\ge N$.
        
        \item For $m>n$ and $n=N$
        $$
        \sqrt{m} - \sqrt{n}
        =\sqrt{m} - \sqrt{n} \cdot
        \frac{\sqrt{m}+\sqrt{n}}{\sqrt{m}+\sqrt{n}}
        = \frac{m-n}{\sqrt{m}+\sqrt{n}} 
        \geq \frac{m-n}{2\sqrt{m}}
        = \frac{\sqrt{m}}{2} - \frac{n}{2\sqrt{m}}
        $$
        
        $$
        > \frac{1}{2} - \frac{n}{2\sqrt{m}}
        = \frac{1}{2} - \frac{N}{2\sqrt{m}}
        > \frac{1}{2} > \epsilon
        $$
        
        Since $m$ as an enough large number, $\frac{N}{2\sqrt{m}}$ would converge to 0 (where $N$ is a fixed natural number).
        Thus we proved that $(x_n)$ is not a Cauchy sequence.
    \end{itemize}
\end{enumerate}


\newpage
\textbf{Question 5.3}
\begin{enumerate}[label=(\roman*)]
    \item Construct a sequence $(x_n)$ in $\mathbb{R}$ such that the set of subsequential limits of $(x_n)$ is the interval [0,1]. Is it possible to construct a sequence where the set of subsquential limits is (0,1)?\\
    
    Consider a sequence $(x_n)$
    $\begin{cases}
        x_{2^n} = 0\\
        x_{2^n+k} = \frac{k}{2^n}
    \end{cases}$
    \begin{itemize}
        \item $2^1 = 0$
        \item $2^2 = 0,\;\;2^3 = \frac{1}{2},\;$
        \item $2^4 = 0,\;\;2^5 = \frac{1}{4},\;\;2^6 = \frac{2}{4},\;\;2^7 = \frac{3}{4}$
        \item $2^8 = 0,\;\;2^9 = \frac{1}{8},\;\;2^{10} = \frac{2}{8},\;\;2^{11} =              \frac{3}{8},\;\;2^{12}=\frac{4}{8},\;\;\ldots$
    \end{itemize}
    We can observe that the sequence $(x_n)$ is in the interval $[0,1]$.
    
    \vspace{1.5\baselineskip}
    \item Let $E = \mathbb{Q} \cap [0,1]$. Find a sequence in $E$ that has no subsequence converging to a point of $E$. 
    \begin{itemize}
        \item Consider $x_n = \frac{1}{\left(1+\frac{1}{n}\right)^n}$ which is contained in $\in E$.
        
        \item As $n\to\infty,\; x_n\to \frac{1}{e}$ due to the fact that $e=\left(1+\frac{1}{n}\right)^n$.
        
        \item Since $e\in\mathbb{R}\setminus\mathbb{Q},\;\frac{1}{e}\in\mathbb{R}\setminus\mathbb{Q}$.
    \end{itemize}
\end{enumerate}


\newpage
\textbf{Question 5.4} \\
Let $(x_n)$ and $(y_n)$ be sequences of real numbers. Prove that
$$ 
\limsup\limits_{n\rightarrow\infty} (x_n + y_n) \le
\limsup\limits_{n\rightarrow\infty} (x_n) + \limsup\limits_{n\rightarrow\infty} (y_n).
$$
\begin{itemize}
    \item We are considering the cases where both 
    $\limsup\limits_{n\rightarrow\infty} (x_n),\;ㅌ \limsup\limits_{n\rightarrow\infty} (y_n)$ go to $\infty$.
    
    \vspace{0.05\baselineskip}
    
    i.e. $\exists$ an upper bound for $x_n,\; y_n$.
    
    \item Otherwise, in case where $\limsup\limits_{n\rightarrow\infty} (x_n) = +\infty$ and $\limsup\limits_{n\rightarrow\infty} (y_n) = -\infty$,\\
    such that $x_n = (-1)^{2n},\; y_n=(-1)^{2n-1}$, $\limsup\limits_{n\rightarrow\infty} (x_n) + \limsup\limits_{n\rightarrow\infty} (y_n)=0$.
    
    \item Now assume that 
    $x =\limsup\limits_{n\rightarrow\infty} (x_n) \in(-\infty,\infty),\; 
    y=\limsup\limits_{n\rightarrow\infty} (y_n) \in(-\infty,\infty).$\\
    %i.e. $x+y= \limsup\limits_{n\rightarrow\infty} (x_n) + \limsup\limits_{n\rightarrow\infty} (y_n).$
    
    \item Then consider a subsequence of the natural numbers $\{n_k\}$ and $p\in\mathbb{R}$ such that 
    $$\lim_{k \to \infty} (x_{n_k} +y_{n_k})\rightarrow p$$
    
\item Consider another subsequence of the natural number $\{n_{k_l}\}$ and $q,\;r\in\mathbb{R}$ such that 

$$\lim_{k \to \infty} x_{n_{k_l}}\rightarrow q,\;\;
\lim_{k \to \infty} y_{n_{k_l}}\rightarrow r
$$

\item Then we obtain $p=q+r$
\item Since $x_{n_{k_l}}$ is a subsequence of $x_{n_k}$ and $x_{n_k}$ is  a subsequence of $x_n$,
$$
q+r = p \le
\limsup\limits_{n\to\infty} x_n+
\limsup\limits_{n\to\infty} y_n
$$

Thus we proved that 
$$
\limsup\limits_{n\to\infty} (x_n + y_n) \le
\limsup\limits_{n\to\infty} x_n+
\limsup\limits_{n\to\infty} y_n
$$
\end{itemize}

\vspace{1.5\baselineskip}
\textbf{Question 5.5} \\
Construct sequences $(x_n), (y_n)$ of real numbers such that
$$
\limsup\limits_{n\rightarrow \infty} x_n = 
\limsup\limits_{n\rightarrow \infty} y_n = 1, \quad
\limsup\limits_{n\rightarrow \infty} (x_n +y_n) = 0
$$

\vspace{1.5\baselineskip}
Suppose\\
\hspace*{\fill}
{$x_n = \{-1,1,-1,1,-1,\ldots\}$} \hfill {$y_n = \{1,-1,1,-1,1,\ldots\}$}
\hspace*{\fill}

\vspace{1.5\baselineskip}
Then 
$$
\limsup\limits_{n\rightarrow \infty} x_n = 
\limsup\limits_{n\rightarrow \infty} y_n = 1, \text{but }
\limsup\limits_{n\rightarrow \infty} (x_n +y_n) = 0
$$

\newpage
\textbf{Question 5.6}\\
Find $\limsup\limits_{n\rightarrow \infty} x_n$ and $\limsup\limits_{n\rightarrow \infty} x_n$ for the sequence $(x_n)$ defined by $x_1 = 0$ and, for $m\le 1$,
$$
x_{2m} = \frac{x_{2m-1}}{2}, \quad
x_{2m+1} = \frac{1}{2} + x_{2m}.
$$

\end{document}
