% --------------------------------------------------------------
% This is all preamble stuff that you don't have to worry about.
% Head down to where it says "Start here"
% --------------------------------------------------------------
 
\documentclass[12pt]{article}
 
\usepackage[margin=1in]{geometry} 
\usepackage{amsmath,amsthm,amssymb,scrextend}
\usepackage{fancyhdr}
\usepackage{enumitem}
\usepackage{amsmath}
\usepackage{amssymb}
\usepackage{textcomp}
\usepackage{fancybox}
\usepackage{tikz}
\usepackage{tasks}
\pagestyle{fancy}


\newcommand{\N}{\mathbb{N}}
\newcommand{\Z}{\mathbb{Z}}
\newcommand{\I}{\mathbb{I}}
\newcommand{\R}{\mathbb{R}}
\newcommand{\Q}{\mathbb{Q}}
\renewcommand{\qed}{\hfill$\blacksquare$}
\let\newproof\proof
\renewenvironment{proof}{\begin{addmargin}[1em]{0em}\begin{newproof}}{\end{newproof}\end{addmargin}\qed}
% \newcommand{\expl}[1]{\text{\hfill[#1]}$}
\setlength{\parindent}{0pt}
\newenvironment{theorem}[2][Theorem]{\begin{trivlist}
\item[\hskip \labelsep {\bfseries #1}\hskip \labelsep {\bfseries #2.}]}{\end{trivlist}}
\newenvironment{lemma}[2][Lemma]{\begin{trivlist}
\item[\hskip \labelsep {\bfseries #1}\hskip \labelsep {\bfseries #2.}]}{\end{trivlist}}
\newenvironment{problem}[2][Problem]{\begin{trivlist}
\item[\hskip \labelsep {\bfseries #1}\hskip \labelsep {\bfseries #2.}]}{\end{trivlist}}
\newenvironment{exercise}[2][Exercise]{\begin{trivlist}
\item[\hskip \labelsep {\bfseries #1}\hskip \labelsep {\bfseries #2.}]}{\end{trivlist}}
\newenvironment{reflection}[2][Reflection]{\begin{trivlist}
\item[\hskip \labelsep {\bfseries #1}\hskip \labelsep {\bfseries #2.}]}{\end{trivlist}}
\newenvironment{proposition}[2][Proposition]{\begin{trivlist}
\item[\hskip \labelsep {\bfseries #1}\hskip \labelsep {\bfseries #2.}]}{\end{trivlist}}
\newenvironment{corollary}[2][Corollary]{\begin{trivlist}
\item[\hskip \labelsep {\bfseries #1}\hskip \labelsep {\bfseries #2.}]}{\end{trivlist}}
 
 
\begin{document}
\settasks{
counter-format=(tsk[r]),
label-width=4ex
}
% --------------------------------------------------------------
%                         Start here
% --------------------------------------------------------------

\lhead{Math 521}
\chead{Homework 4}
\rhead{Meenmo Kang}


\textbf{Questions 4.1}\\
Let $(X,d)$ be a metric space. Prove that a sequence $(x_n)$ in $X$ converges $x_n \rightarrow x \in X$ if and only if every subsequence $x_{n_k} \rightarrow x.$

\begin{itemize}
    \item $\Rightarrow$ If every subsequence $x_{n_k} \rightarrow x$, a sequence $(x_n)$ in $X$ converges $x_n \rightarrow x \in X$ 
    \begin{itemize}
        \item If every subsequence converges $x_{n_k} \rightarrow x$, $\exists\; K$ such that $k\ge K$ and $n_k \ge N$.  
        
        \item So $|x_{n_k} - x| < \epsilon$.
        
        \item Therefore $\exists\; N $ such that $n\ge N$ and further $x_n$ such that $|x_n - x| <\epsilon$.
        
        \item Thus $lim_{k\rightarrow \infty} \; x_{n} = x$.
    \end{itemize}
    
    \item $\Leftarrow$ If a sequence $(x_n)$ in $X$ converges $x_n \rightarrow x \in X$, every subsequence $x_{n_k} \rightarrow x$ .
    \begin{itemize}
        \item If a sequence $(x_n)$ in $X$ converges $x_n \rightarrow x \in X$,\\ $\exists\; N$ such that $n\ge N$ implying $|x_n - x| < \epsilon$.
        
        \item Then $\exists\; K$ such that $k\ge K$ and $n_k \ge N$. 
        
        \item Hence $\exists$ a subsequence $x_{n_k}$ such that $|x_{n_k} - x| < \epsilon$ also.
        
        \item Thus $lim_{k\rightarrow \infty} \; x_{n_k} = x$.
    \end{itemize}
    
\end{itemize}

\textbf{Question 4.2}\\
Let $(X,d)$ be a metric space and let $(x_n)$ be a sequence in $X$. Prove $x_n \rightarrow x\in X$ if and only if $d(x_n,x) \rightarrow 0$ in $\mathbb{R}$.


\begin{itemize}
    \item $\Leftarrow$ If $x_n \rightarrow x\in X$, $d(x_n,x) \rightarrow 0$ in $\mathbb{R}$.
    \begin{itemize}
        \item If $x_n \rightarrow x\in X, \exists\; N$ such that $n\ge N$ and $|x_n - x| < \epsilon$ where $\epsilon > 0$.
        
        \item So we obtain $-\epsilon < |x_n - x| < \epsilon$. 
        
        \item Since we chose an arbitrary $\epsilon$, it can be as small as possible as long as it is greater than 0. 
        
        \item Take $lim_{n\rightarrow \infty} -\frac{\epsilon}{n} < |x_n - x| < \frac{\epsilon}{n}$.
        
        \item Thus $d(x_n, x)$ converges to 0 eventually.
    \end{itemize}
    
    \item $\Rightarrow$ If $d(x_n,x) \rightarrow 0$ in $\mathbb{R}$, $x_n \rightarrow x\in X$.
    \begin{itemize}
        \item Since $d(x_n,x)$ is converging to 0, we cannot say it is equal to 0.\\ 
        i.e. it is greater than 0.  
        
        \item Then consider $\epsilon > 0$ such that $0 < d(x_n, x) = |x_n - x| < \epsilon$.

        \item Then $\exists\; N$ such that $n\ge N$.
        
        \item Thus $x_n \rightarrow x \in X$.
    \end{itemize}
\end{itemize}

\newpage
\textbf{Question 4.3}\\
Let $(X,d)$ be a metric space, $C\subset X$.
\begin{enumerate}[label=(\roman*)]
    \item Let $C$ be closed, $(x_n)$ a sequence in $C$. Prove that if $x_n \rightarrow x \in X,$ then $x\in C$.
    \begin{itemize}
        \item Suppose by contradiction that $x_n \in C$, but $x\in X \setminus C$ where is open.
        
        \item By definition of open, $\exists\; B_\epsilon(x) \subset X\setminus C$. 
        \\i.e. the ball $B_\epsilon(x)$ is completely contained in $X\setminus C$.
        
        \item Hence $\exists\; N$ such that $n\ge N, x_n\in B_\epsilon(x)$.
        
        \item This leads to $x_n \subset X\setminus C$. However the sequence $x_n$ is in $C$.

        \item Thus this contradicts to the assumption we made and we proved that \\
        if $x_n \rightarrow x \in X,$ then $x\in C$ as required.
    \end{itemize}
    
    \item Suppose that for every sequence $(x_n)$ in $C$ such that $x_n \rightarrow x \in X$ that $x \in C$. Prove that $C$ is closed.
    \begin{itemize}
        \item $C\subset X$ and $\exists$ a limit point $x$ of $C$.
        \item This implies that $\exists$ a sequence $x_n$ that converges to $x$ in $C$ also.
        \item Since the given limit point $x$ and $x_n$ are proven to be in the set $C$, $C$ is a closed set.
    \end{itemize}
    
\end{enumerate}

\textbf{Question 4.4}\\
Let $A\subset \mathbb{R}$ and suppose $(x_N)$ is a sequence of upper bounds for $A$, that is, suppose that for every $n\in \mathbb{N},\; x_n$ is an upper bound for $A$. Suppose $x_n \rightarrow x$. Prove that $x$ is an upper bound for $A$.
\begin{itemize}
    \item $x_n$ is an upper bound of A. Then $\exists\; a\in A$ such that $a \le min\{x_n\} \le x_n$.
    
    \item Since $x_n \rightarrow x$, $\exists\; N\in \mathbb{N}$ such that $n\ge N$ and $|x_n - x| < \epsilon \;\; \forall \;\epsilon > 0$.
    
    \item i.e. $-\epsilon < x_n - x < \epsilon \Leftrightarrow x - \epsilon < x_n < x + \epsilon$\\
    
    \item Suppose by contradiction that $x+\epsilon  < supA\;\;\forall\;\epsilon>0$. Then $x_n < x + \epsilon < supA$. 
    \item Since $x_n$ is an upper bound of $A$, $supA \le x_n$.
    \item And by the inequality we obtained above, $supA\le x_n < x+\epsilon'$ which contradicts with our assumption.
    \item Hence $x$ is also an upper bound of $A$.
\end{itemize}

\newpage
\textbf{Question 4.5}\\
Let $(x_n), (y_n), (z_n)$ be sequences of real numbers such that $x_n \rightarrow y, z_n \rightarrow y$ and, for all $n\in \mathbb{N},\; x_n \le y_n \le z_n.$ Prove that $y_n \rightarrow y$ also.

\vspace{1.5\baselineskip}
Take $\epsilon>0$ and suppose $\exists\; N',N'' \in \mathbb{N}$ such that
\begin{tasks}(2)
    \task $|x_n - y| < \epsilon$ for $n\ge N'$\\
    $\Rightarrow y-\epsilon < x_n < y + \epsilon$
    \task $|z_n - y| < \epsilon$ for $n\ge N''$\\
    $\Rightarrow y-\epsilon < z_n < y + \epsilon$
\end{tasks}
\begin{itemize}
    \item Since $ x_n \le y_n \le z_n$,\; $y-\epsilon < x_n \le y_n \le z_n < y+\epsilon$.
    \item Hence $y-\epsilon < y_n < y+\epsilon$. i.e. $|y_n - y| < \epsilon$.
    \item Thus $y_n$ converges to $y$ also.
\end{itemize}


\textbf{Question 4.6}\\
For a sequence $(x_n)$ of real numbers, define the arithmetic mean $\sigma_n$ by 
$$\sigma_n = \frac{x_1+\ldots + x_n}{n}.
$$
\begin{enumerate}[label=(\roman*)]
    \item Prove that if $lim_{n\rightarrow \infty} x_n = x,$ then $lim_{n\rightarrow \infty} \sigma_n = x$ .
    \item Prove by a counterexample that the converse statement is not true. More precisely, construct a sequence $(x_n)$ that does not converge such that $\sigma_n \rightarrow 0$.
\end{enumerate}



\end{document}
