% --------------------------------------------------------------
% This is all preamble stuff that you don't have to worry about.
% Head down to where it says "Start here"
% --------------------------------------------------------------
 
\documentclass[12pt]{article}
 
\usepackage[margin=1in]{geometry} 
\usepackage{amsmath,amsthm,amssymb,scrextend}
\usepackage{fancyhdr}
\usepackage{enumitem}
\usepackage{amsmath}
\usepackage{amssymb}
\usepackage{textcomp}
\usepackage{fancybox}
\usepackage{tikz}
\usepackage{cancel}
\usepackage{tasks}
\pagestyle{fancy}


\newcommand{\N}{\mathbb{N}}
\newcommand{\Z}{\mathbb{Z}}
\newcommand{\I}{\mathbb{I}}
\newcommand{\R}{\mathbb{R}}
\newcommand{\Q}{\mathbb{Q}}
\renewcommand{\qed}{\hfill$\blacksquare$}
\let\newproof\proof
\renewenvironment{proof}{\begin{addmargin}[1em]{0em}\begin{newproof}}{\end{newproof}\end{addmargin}\qed}
% \newcommand{\expl}[1]{\text{\hfill[#1]}$}
\setlength{\parindent}{0pt}
\newenvironment{theorem}[2][Theorem]{\begin{trivlist}
\item[\hskip \labelsep {\bfseries #1}\hskip \labelsep {\bfseries #2.}]}{\end{trivlist}}
\newenvironment{lemma}[2][Lemma]{\begin{trivlist}
\item[\hskip \labelsep {\bfseries #1}\hskip \labelsep {\bfseries #2.}]}{\end{trivlist}}
\newenvironment{problem}[2][Problem]{\begin{trivlist}
\item[\hskip \labelsep {\bfseries #1}\hskip \labelsep {\bfseries #2.}]}{\end{trivlist}}
\newenvironment{exercise}[2][Exercise]{\begin{trivlist}
\item[\hskip \labelsep {\bfseries #1}\hskip \labelsep {\bfseries #2.}]}{\end{trivlist}}
\newenvironment{reflection}[2][Reflection]{\begin{trivlist}
\item[\hskip \labelsep {\bfseries #1}\hskip \labelsep {\bfseries #2.}]}{\end{trivlist}}
\newenvironment{proposition}[2][Proposition]{\begin{trivlist}
\item[\hskip \labelsep {\bfseries #1}\hskip \labelsep {\bfseries #2.}]}{\end{trivlist}}
\newenvironment{corollary}[2][Corollary]{\begin{trivlist}
\item[\hskip \labelsep {\bfseries #1}\hskip \labelsep {\bfseries #2.}]}{\end{trivlist}}
 
 
\begin{document}
\settasks{
counter-format=(tsk[r]),
label-width=4ex
}
% --------------------------------------------------------------
%                         Start here
% --------------------------------------------------------------

\lhead{Math 521}
\chead{Homework 7}
\rhead{Meenmo Kang}


\textbf{Question 7.1}\\
Let $f\colon X\rightarrow \mathbb{R}$ be a continuous function. Prove that the zero set of $f$, that is the set 
$$\{x\inX\;|\;f(x)=0\}$$ is closed.
\begin{itemize}
    \item Note that the Corollary of Theorem 4.8 on Rudin states {\sl A mapping f of a metric space X into a metric space Y is continuous if and only if $f^{-1}(C)$ is closed in X for every closed set C in Y.}
    \item Suppose $g(f) = f^{-1}(\{0\})$. By the Corollary above, $g(f)$ is closed if $f$ is continuous.\\
    
    \item Let $x$ be a limit point of $g(f)$. Then $\exists$ a sequence $\{x_n\}$ in $g(f)$ such that $d(x_n,x)\rightarrow 0$.
    \item $f$ is continuous at $x$. So $\exists\; \delta>0\;\;\forall \epsilon>0$ such that $d(x_n,x)<\delta \Leftrightarrow d(f(x_n)-f(x))<\epsilon$.
    \item As $\{x_n\}$ converges to $x$, $\exists\; N$ such that $n\ge N$; hence $d(f(x_n)-f(x))<\epsilon$.
    \item Since $f(x_0)=0,\; d(f(x_n)-f(x)) = d(f(x)) < \epsilon \;\;\forall\; \epsilon>0.$
    \item This leads to $f(x) = 0$, or $x\in g(f)$, and $g(f)$ is, therefore, closed since it contains all its limit points.
    
\end{itemize}


\vspace{1.5\baselineskip}
\textbf{Question 7.2}\\
We say that a function $f\colon X\rightarrow Y$ is Lipschitz if there exists a real number $L>0$ such that 
$$d(f(x_1),f(x_2))\le Ld(x_1,x_2)$$ for all $x_1,x_2\in X$.
Show that every Lipshitz function is continuous.
\begin{itemize}
    \item Fix $c\in X$, and let $x_1\in X$ converges to $c$ and $\epsilon>0$ be given.
    \item Suppose $\exists\; N$ such that $n>N$ and $d(x_1 - c) \le \epsilon / L$.
    \item Now we hold, $$d(f(x_1)-f(c)) \le Ld(x_1,c)\le \epsilon$$
    \item i.e. $f(x_1)$ converges to $f(c)$.
    \item The continuity of $f$ on $X$ is proved.
\end{itemize}


\newpage                                                                                          
\textbf{Question 7.3}\\
Let $p\in X$ be a fixed point. Prove that the function $f(x)=d(x,p)$ is Lipschitz, and hence continuous.
\begin{itemize}
    \item Suppose $x,y\in X$. So $f(x) = d(x,p) \le d(x,y)+d(y,p) = d(x,y)+ f(y)$
    \item $\Leftrightarrow f(x)-f(y) \le d(x,y) \Leftrightarrow f(y)-f(x) \le d(y,x)=d(x,y)
    \Leftrightarrow |f(x),f(y)| \le d(x,y)$
    \item As there exists Lipshitz constant which is 1, $$|f(x)-f(y)|\le 1\cdot d(x,y),$$ $f(x)$ is Lipschitz and, hence continuous.
    
\end{itemize}

\vspace{1.5\baselineskip}
\textbf{Question 7.4}\\
Let $E\subset X$ and define a function $f\colon X\rightarrow \mahttbb{R}$ by $$f(x) = \inf\{d(x,p)\;|\;p\in E\}.$$
Show that $f$ is Lipschitz, and hence continuous.
\begin{itemize}
    \item Suppose $x,y\in X$. 
    \item So $f(x) = \inf\{d(x,p)\;|\;p\in E\}\le d(x,p) \le d(x,y)+d(y,p).$
    \item And the by definition of $f$, $f(y) = \inf\{d(y,p)\;|\;p\in E\}\le d(y,p).$
    \item Then $f(x)-f(y) \le \{d(x,y)+d(y,p)\}-d(y,p) = d(x,y) \Leftrightarrow f(y)-f(x) \le d(y,x).$
    \item Hence $|f(x)-f(y)| \le 1\cdot d(x,y)$.
    \item The proper Lipschitz constant of $f$ which is 1, and its continuity are proved.
    
\end{itemize}


\vspace{1.5\baselineskip}
\textbf{Question 7.5}\\
Let $K\subset X$ be compact and suppose that $x\in X\setminus K$. Prove that there exists a nearest point of $K$ to $x$, that is, that there exists $p\in K$ such that $d(x,p)\le d(x,q)$ for all $q\in K$.

\vspace{1.5\baselineskip}
\textbf{Question 7.6}\\
Define the function $f\colon\mathbb{R}\rightarrow\mathbb{R}$ by $f(x)=\sqrt{|x|}$. Is $f$ Lipschitz?
\begin{itemize}
    \item Suppose that $x,y\in \mathbb{R}$. To prove whether $f$ is Lipschitz, we need to find a Lipschitz constant $C$ such that $d(\sqrt{x}-\sqrt{y})\le Cd(x,y)$ which holds on any interval.
    \item Fix $y=0$. Then we obtain 
    $$\frac{\sqrt{x}-\sqrt{0}}{|x-0|} = \frac{1}{\sqrt{x}} \le C <1$$
    \item However, $1/\sqrt{x} < 1$ does not hold for all $x$. Thus $f$ is not Lipschitz.
\end{itemize}
\vspace{1.5\baselineskip}
\textbf{Question 7.7}\\
Let $f,g: X\rightarrow Y$ be continuous functions and suppose that $E\subset X$ is dense. Prove that $f(E)$ is dense in $f(X)$. Suppose now that $f(p) = g(p)$ for all $p\in E$. Prove that $f(x)=g(x)$ for all $x\in X$.


\end{document}
