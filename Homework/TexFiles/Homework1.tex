\documentclass[12pt]{article}
 
\usepackage[margin=1in]{geometry} 
\usepackage{amsmath,amsthm,amssymb,scrextend}
\usepackage{fancyhdr}
\usepackage{enumitem}
\pagestyle{fancy}

 
\newcommand{\N}{\mathbb{N}}
\newcommand{\Z}{\mathbb{Z}}
\newcommand{\I}{\mathbb{I}}
\newcommand{\R}{\mathbb{R}}
\newcommand{\Q}{\mathbb{Q}}
\renewcommand{\qed}{\hfill$\blacksquare$}
\let\newproof\proof
\renewenvironment{proof}{\begin{addmargin}[1em]{0em}\begin{newproof}}{\end{newproof}\end{addmargin}\qed}
% \newcommand{\expl}[1]{\text{\hfill[#1]}$}
 
\newenvironment{theorem}[2][Theorem]{\begin{trivlist}
\item[\hskip \labelsep {\bfseries #1}\hskip \labelsep {\bfseries #2.}]}{\end{trivlist}}
\newenvironment{lemma}[2][Lemma]{\begin{trivlist}
\item[\hskip \labelsep {\bfseries #1}\hskip \labelsep {\bfseries #2.}]}{\end{trivlist}}
\newenvironment{problem}[2][Problem]{\begin{trivlist}
\item[\hskip \labelsep {\bfseries #1}\hskip \labelsep {\bfseries #2.}]}{\end{trivlist}}
\newenvironment{exercise}[2][Exercise]{\begin{trivlist}
\item[\hskip \labelsep {\bfseries #1}\hskip \labelsep {\bfseries #2.}]}{\end{trivlist}}
\newenvironment{reflection}[2][Reflection]{\begin{trivlist}
\item[\hskip \labelsep {\bfseries #1}\hskip \labelsep {\bfseries #2.}]}{\end{trivlist}}
\newenvironment{proposition}[2][Proposition]{\begin{trivlist}
\item[\hskip \labelsep {\bfseries #1}\hskip \labelsep {\bfseries #2.}]}{\end{trivlist}}
\newenvironment{corollary}[2][Corollary]{\begin{trivlist}
\item[\hskip \labelsep {\bfseries #1}\hskip \labelsep {\bfseries #2.}]}{\end{trivlist}}
 
\begin{document}
 
% --------------------------------------------------------------
%                         Start here
% --------------------------------------------------------------

\lhead{Math 521}
\chead{Homework 1}
\rhead{Meenmo Kang (H)}

\noindent \textbf{Question 1.1}\\
Let $F = {0,1}$ be the set called the paddock, defined in class, and define addition and multiplication on $F$ by 
$$0 + 0 = 0,\quad 0 + 1 = 1 + 0 = 1,\quad 1 + 1 = 0,$$
and
$$ 0 \cdot 0 = 0\cdot1 = 1\cdot0 = 0, \quad 1\cdot1 = 1$$
\begin{itemize}
    \item Is $F$ a field? 
        \begin{itemize}
            \item Check if addition is closed.
                \begin{itemize}
                    \item (A1) 0 + 1 = 0 $\in F$
                    \item (A2) 0 + 1 = 1 + 0 = 1
                    \item (A3)
                        \begin{itemize}
                            \item (0 + 1) + 1 = 1 + 1 = 0\\
                                  0 + (1 + 1) = 0 + 0 = 0\\
                                  Thus (0 + 1) + 1 = 0 + (1 + 1) \\
                            \item (0 + 1) + 0 = 1 + 0 = 1\\
                                  0 + (1 + 0) = 0 + 1 = 1\\
                                  Thus (0 + 1) + 0 = 0 + (1 + 0) 
                        \end{itemize}
                        
                    \item (A4)
                        \begin{itemize}
                            \item 0 + 1 = 1 $\in F$
                            \item 0 + 0 = 0 $\in F$
                        \end{itemize}
                    \item (A5) 
                        \begin{itemize}
                            \item 0 + (-0) = 0
                            \item 1 + (-1) = 0
                        \end{itemize}
                    \item Therefore $F$ is closed under addition.
                \end{itemize}
                
            \item Check if multiplication is closed.
                \begin{itemize}
                    \item (M1) $0 \cdot 1 = 0 \in F$
                    \item (M2) $0 \cdot 1 = 1 \cdot 0 = 0 $
                    \item (M3) 
                        \begin{itemize}
                            \item $(1\cdot 0)\cdot 0 = 1\cdot (0\cdot 0) = 0$
                            \item $(1\cdot 0)\cdot 1 = 1\cdot (0\cdot 1) = 0$
                        \end{itemize}
                    \item (M4)
                        \begin{itemize}
                            \item $1\cdot0=0$
                            \item $1\cdot1=1$
                        \end{itemize}
                    \item (M5)
                        \begin{itemize}
                            \item $\frac{1}{1}\cdot1=1$
                        \end{itemize}
                    \item Therefore $F$ is closed under multiplication.
                \end{itemize}
        \end{itemize}
    \textbf{Thus F is a field.}
    \item Is $F$ an ordered field?
        \begin{itemize}
            \item In case where $1 > 0$
            \item $1 + 1 > 1 + 0$
            \item $0 > 1$, which is a contradiction
        \end{itemize}
\end{itemize}
\textbf{Thus $F$ is not an ordered field}\\
\newline
\noindent \textbf{Question 1.2}\\ 
Prove that there is no rational number $p \in \Q$ such that $2^p = 6$.
\begin{itemize}
    \item Argue by contradiction. Suppose $\exists$ a rational number p such that $2^p = 6$.
    \item If there were such a $p$, we could write $\frac{a}{b}$ where a and b are integers that are not both even.
    \item Then $2^p = 2^{\frac{a}{b}} = 6$. Hence $2^a = 6^b = (2\cdot 3)^b = 2^b \cdot 3^b \; \Leftrightarrow \; \mathbf{2^{a-b} = 3^b}.$
    \item Since $6 > 2^1$, $p=\frac{a}{b}>1$, which means $a > b$.
    \item Thus $2^{a-b} > 2$, and $2^{a-b}$ must be even.
    \item On the other hand, $3^b$ must be odd.
    \item i.e. $2^{a-b} = 3^b$ is impossible.
    \item Thus there is no such $p$ that satisfies the condition $2^p=6$.
\end{itemize}
\noindent \textbf{Question 1.3}\\
Use the Archimedean property to prove that for any pair of rational numbers $p < q$, there exists an irrational number x such that $p < x < q$.\\
\textbf{Hint}: You may use without proof the fact that if $r \neq 0$ is rational and x is irrational, then $r + x$ and $rx$ are both irrational.
\begin{itemize}
    \item We are given $p < q$, or equivalently $q - p > 0$. 
    \item According to the Archimedean property, there is a positive integer n, such that\\ $n(q - p) > 2\;$(instead of 1). 
    \item Let $x = \frac{q-p}{2}$, and clearly we obtain $p < x < q$, and $\frac{q-p}{2} > \frac{1}{n} \Leftrightarrow \frac{1}{n} < x$.
    \item Hence $p < x + \frac{1}{n} < 2x = q - p < q$.
    \item Consider irrational $\frac{1}{n\sqrt{2}}$ in lieu of $\frac{1}{n}$.
    \item Since $\frac{1}{n\sqrt{2}}$ is still positive, it preserves the inequalities $p < x + \frac{1}{n\sqrt{2}} < q$.
    \item Since the addition of rational $x$ and irrational $\frac{1}{n\sqrt{2}}$ is irrational number, there exists an irrational such that $p < x + \frac{1}{n\sqrt{2}} < q$.
\end{itemize}
\noindent \textbf{Question 1.4}\\
Let $A \subset \R$ be non-empty and bounded below. Define $-A = \{-a\;|\;a \in A\}$. Prove that $$\text{inf A = - sup(-A)}$$
\begin{itemize}

    \item inf A $\le$ -sup(-A)
        \begin{itemize}
            \item Since inf A is an infimum for A, inf A $\le a\;\; \forall a \in A$, \item Then -inf A $\ge -a  \;\; \forall -a \in -A.$
            \item In other words, -inf A is an upper bound for -A.
            \item Hence sup(-A) $\le$ -inf A $\Leftrightarrow$ inf A $\le$ -sup(-A)
        \end{itemize}
    \item inf A $\ge$ -sup(-A)
        \begin{itemize}
            \item Since A is bounded below, -A is bounded above.
            \item Then $-a \le$ sup(-A) $\Leftrightarrow a \ge$ -sup(-A)
            \item This means that -sup(-A) is a lower bound for A implying inf(A) $\ge$ -sup(-A).
        \end{itemize}

\end{itemize}
\textbf{inf A $\le$ -sup(-A) and inf A $\ge$ -sup(-A) at the same time. Thus inf A = -sup(-A)}\\

\noindent \textbf{Question 1.5}\\
Let $A,B \subset \R$ be non-empty and bounded subsets of the real numbers, and let $\lambda \in \R$ be a constant.

\begin{enumerate}[label=(\alph*)]
    \item Let $\lambda A = \{\lambda A \; | \; a \in A \}.$ Prove that if $\lambda \ge 0$, then sup$\lambda A = \lambda$sup A, and if $\lambda < 0$, then sup $\lambda A = \lambda$inf A. 
    
    \begin{enumerate}[label=\roman*.]
        \item $\lambda \ge 0$
            \begin{itemize}
                \item In case where $\lambda = 0$, its supremum and infimum are both 0.
                \item Fix $\epsilon > 0$, and suppose $\exists\; \alpha \in A$ such that sup A - $\epsilon < \alpha \le$ sup A by definition of supremum.
                \item Multiply by $\lambda$ each side of the inequality above \\
                      $$\lambda \text{sup A} - \lambda\epsilon < \lambda\alpha \le \lambda \text{sup A}.$$
                \item Hence $\lambda \text{sup A}$ is a supremum of $\lambda\alpha$. At the same time, sup $\lambda$A is also a supremum of $\lambda\alpha$ since $\lambda\alpha \in \lambda$A.
                \item Thus sup $\lambda$A $= \lambda$sup A
            \end{itemize}
        
        \item $\lambda < 0$
            \begin{itemize}
                \item Consider the inequalities we stated above sup A - $\epsilon < \alpha \le$ sup A.
                \item Multiply this inequalities by $\lambda$
                      $$\lambda \text{sup A} \le \lambda\alpha <  \lambda \text{sup A} - \lambda\epsilon$$
                \item Since we proved that sup $\lambda$A $= \lambda$sup A, we obtain $$\text{sup $\lambda$A} \le \lambda\alpha \;\Leftrightarrow\; \alpha \ge \frac{1}{\lambda}\cdot\text{sup $\lambda$A}$$
                \item This means that inf $\alpha = \frac{1}{\lambda}\cdot\text{sup $\lambda$A}$. 
                \item Hence $\lambda\text{inf }\alpha = \text{sup $\lambda$A}$
            \end{itemize}
    \end{enumerate}
    
    \item Let $A + B = \{a+b \;|\; a \in A, \;b \in B \}.$ Prove sup(A + B) = sup A + sup B.
        \begin{itemize}
            \item By definition, sup A $\ge$ a, and sup B $\ge$ b.
            \item Thus \textbf{sup A + sup B $\ge$ a + b}.
            \item Suppose $\epsilon > 0$, and consider a and b such that a $>$ sup A $- \;\frac{\epsilon}{2}$, b $>$ sup B $-\; \frac{\epsilon}{2}$.
            \item Then \textbf{a + b $>$ sup A + sup B - $\epsilon$}.
            \item Thus sup(A + B) $\ge$ a + b = sup A + sup B - $\epsilon$.\\
            \item Suppose x such that $x \in A + B$
            \item $x \le \text{sup A + sup B}$
            \item sup x $\le$ sup A + sup B $\Leftrightarrow$ sup (A + B) $\le$ sup A + sup B
        \end{itemize}
        We obtain both sup(A + B) $\ge$ = sup A + sup B and sup (A + B) $\le$ sup A + sup B. Thus sup(A + B) = sup A + sup B. 
    \item Let AB = $\{a\cdot b \;|\; a \in A, \;b \in B \}.$ Either prove or provide a counterexample to sup(AB) = (sup A)(sup B).
        \begin{itemize}
            \item Assume $\exists$ A = $\{1,2,3\}$, B = $\{-1,-2,-3\}$. Then AB = $\{-9,-6,-4,-3,-2,-1\}$.
            \item sup AB = -1, sup A = 3, sup B = -1 respectively.
            \item sup A $\cdot$ sup B = -3 which is not equal to sup AB = -1.
        \end{itemize}
\end{enumerate}


\noindent \textbf{Question 1.6}\\
Let $A \subset \R$ be non-empty and bounded above. Prove that $\alpha = \text{sup A}$ if and only if $x \le \alpha$ for every $x \in A$ and for every $\epsilon > 0$, there exists $x \in A$ with $\alpha - \epsilon < x \le \alpha$.
\begin{itemize}

    \item $\alpha = \text{sup A}$ \textbf{if} $\alpha = \text{sup A}$ if and only if $x \le \alpha$ for every $x \in A$ and for every $\epsilon > 0$, there exists $x \in A$ with $\alpha - \epsilon < x \le \alpha$.
        \begin{itemize}
            \item It is obvious that $\alpha$ is an upper bound for A $\forall x \le \alpha$.
            \item Argue by contradiction. Suppose there exists $\beta$ such that sup A = $\beta$.
            \item i.e. $\beta < \alpha$ and $x \le \beta \;\; \forall x \in A \Leftrightarrow 0 < \alpha - \beta$.
            \item Suppose $\exists\; \epsilon\;\; \text{such that}\;\; \epsilon = \frac{\alpha - \beta}{2} > 0$
            \item i.e. $\alpha - \epsilon < \alpha$
            \item Hence $\beta < \alpha \Leftrightarrow 2\beta < \alpha + \beta \Leftrightarrow \beta < \frac{\alpha + \beta}{2} = \epsilon \Leftrightarrow \beta < \alpha - \epsilon$
            \item Hence further we obtain $\beta < \alpha - \epsilon < \alpha$.
            \item i.e. $x < \alpha - \epsilon < \alpha \;\; \forall x \in A$, which is contradicted by the given condition that there exists $x \in A$ with $\alpha - \epsilon < x \le \alpha$.
            \item Thus $\alpha$ must be the least upper bound for A.
        \end{itemize} 
 
    \item $x \le \alpha$ for every $x \in A$ and for every $\epsilon > 0$, there exists $x \in A$ with $\alpha - \epsilon < x \le \alpha$ \textbf{if} $\alpha = \text{sup A}$
    \begin{itemize}
        \item $x \le \alpha$ for every $x \in A$
        \begin{itemize}
            \item Argue by contradiction. Suppose $\exists\; x > \alpha \;\;\forall x \in A$.
            \item By definition, $\alpha$ is not an upper bound of A if $x > \alpha$.
            \item Therefore, sup A must be equal to $\alpha$ such that $x \le \alpha$ for every $x \in A$.
        \end{itemize}
    \item For every $\epsilon > 0$, there exists $x \in A$ with $\alpha - \epsilon < x \le \alpha$
        \begin{itemize}
            \item Argue by contradiction. Suppose $\exists \;\; x \le \alpha - \epsilon$ for some $x$.
            \item i.e. sup A $\neq \alpha$, but sup A = $\alpha - \epsilon$, which is contradiction.
            \item Hence $\alpha - \epsilon < x \le \alpha \;\;\forall x \in A$
        \end{itemize}
    \end{itemize}
    

        
\end{itemize}

\end{document}